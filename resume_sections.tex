% ===== SHARED SECTIONS =====

% Name and Contact Information (Shared)
\newcommand{\NameAndContact}{
  % Center, bold, and enlarge name
  {\centering
  \huge\textbf{Nick Cline}
  \par}

  \vspace{0.5em}  % Add some vertical space

  % Center and reduce spacing for contact info
  {\centering
  \small
  Somerville, MA \textbar{} +1 978-880-3507 \textbar{}
  nicholascline1@gmail.com \textbar{} nickcline.com
  \par}
  
  \vspace{-1.1em}  % Remove extra space below contact info
}

% Education Section (Shared)
\newcommand{\Education}{
  \sectionheading{Education}
  \entry{University of Massachusetts Amherst Amherst, MA}{May 2022}
  \textit{B.S., M.S. in Mechanical Engineering \textbar{} GPA 3.49/4.0}\\
  \textit{Minor in Computer Science}
}

% Publications Section (Shared)
\newcommand{\Publications}{
  \sectionheading{Publications}
  {\small
    Price, M., Abdikadirova, B., Locurto, D., Jaramillo, J. M., \textbf{Cline, N.}, Hoogkamer, W., \& Huber, M. E. (2022). Unilateral stiffness modulation with a robotic hip exoskeleton elicits adaptation during gait. In 2022 IEEE/RSJ International Conference on Intelligent Robots and Systems (IROS) (pp. 12275–12281). IEEE. https://doi.org/10.1109/iros47612.2022.9981067 
  }
}

\newcommand{\MissionStatement}{
  \vspace{-0.5em}  % Reduce space before mission statement
  \begin{center}
    \begin{minipage}{0.9\textwidth}
      \centering
      \large\itshape\color{gray}
      Innovative mechanical engineer with a passion for robotics and control systems, seeking to leverage expertise in AI, machine learning, and mechanism design to push the boundaries of human-robot interaction and create impactful solutions in the field of intelligent systems.
    \end{minipage}
  \end{center}
  \vspace{-0.5em}  % Reduce space after mission statement
}

% Skills & Interests Section (Shared)
\newcommand{\SkillsInterests}{
  \sectionheading{Skills \& Interests}
  \noindent\begin{tabular}{@{}p{5.5em}@{\hspace{0.5em}}p{\dimexpr\textwidth-6em-2\tabcolsep\relax}@{}}
    \begin{minipage}[c][2\baselineskip][c]{5.5em}
      \centering\textbf{Skills:}
    \end{minipage} & 
    Atlassian Suite, C, C++, Python, Rust, Julia, MATLAB, Solidworks, ROS 2, Gazebo, Drake, MPC, Nonlinear Dynamics, OpenCV. \\[0.15em]
    \begin{minipage}[c][2\baselineskip][c]{5.5em}
      \centering\textbf{Interests:}
    \end{minipage} & 
    Robotics, Control Theory, Machine Learning, AI, Mechanism Design, 3D Printing, Gardening, Cooking, Weightlifting.
  \end{tabular}
}

% ===== RESEARCH RESUME SECTIONS =====

% Research Experience Section - Research Version
\newcommand{\ResearchExperienceResearch}{
  \sectionheading{Research Experience}
  \entry{University of Massachusetts Amherst HRSL, Amherst, MA}{Dec 2021 -- May 2022}
  \textit{Graduate Researcher}
  \begin{itemize}
  \item Created control scripts in python to implement custom impedance control loops for a robotic hip exoskeleton using a Raspberri Pi 4 and the FlexSEA API.
  \item Applied asymmetric perturbations to human gait to study behavioral adaptations to a resistant exoskeleton stiffness. Compiled and analyzed data from motion capture and ground reaction forces to identify key trends and possible neuromotor impacts.
  \item Used findings to evaluate the gait patterns resulting from the exoskeleton's impedance with results published in IROS 2022.
  \item Identified new control schemes for gait control, documenting and testing the initial implementation. Takeaways from the experiments were used to develop tools for realtime data analysis and an improved setup process.
  \end{itemize}
}

% Work Experience Section - Research Version
\newcommand{\WorkExperienceResearch}{
  \sectionheading{Work Experience}
  \entry{Draper Laboratory, Cambridge, MA}{Aug 2022 -- Present}
  \textit{Mechanical Instrumentation Engineer II}
  \begin{itemize}
  \item Developed innovative design features for a sensor assembly by exploring trade spaces and proposing new solutions, coordinating with a cross-functional team to plan reviews, track performance and meet strict requirements.
  \item Designed and tested production prototypes, analyzing test data to identify areas for revision and validate production processes for critical assembly features.
  \end{itemize}
}

% Project Experience Section - Research Version
\newcommand{\ProjectExperienceResearch}{
  \sectionheading{Project Experience}
  \entry{Robotic Simulation}{October 2023 -- Present}
  \begin{itemize}
  \item Built a Furuta pendulum model for use in Gazebo and Drake, controlled with C++ scripts and ROS 2. Developed swing-up and energy-minimizing controllers to stabilize pendulum. Future implementations include RL control of the pendulum and double pendulum control.
  \item Created a bipedal robot model in Gazebo and used Julia to output inverse dynamics from the system configuration. Future implementations include full-body control of the bipedal robot.
  \end{itemize}

  \entry{Robotic System Design}{January 2023 -- Present}
  \begin{itemize}
  \item Designed and prototyped cycloidal and planetary gearboxes for use with brushless DC motors. Tested prototypes using speed control and iterated on design to improve assembly and strength.
  \item Produced a benchtop dynamometer prototype to compare performance and characteristics of two competing DC motors. 
  \end{itemize}

  \entry{Personal Website Tools}{June 2024 -- Present}
  \begin{itemize}
  \item Developed modular LaTeX templates to streamline resume generation and standardize formatting across different versions, with live versions hosted on GitHub and viewable on my personal website.
  \item Created JavaScript widgets to display nonlinear dynamical systems, with interactive features to change initial conditions and parameters.
  %\item Developed a counter in Rust to record the number of times a user has clicked the widget.
  \end{itemize}
}

% ===== REGULAR RESUME SECTIONS =====

% Research Experience Section - Resume Version
\newcommand{\ResearchExperienceResume}{
  \sectionheading{Research Experience}
  \entry{University of Massachusetts Amherst HRSL, Amherst, MA}{Dec 2021 -- May 2022}
  \textit{Graduate Researcher}
  \begin{itemize}
  \item Created control scripts in Python to interface with a robotic exoskeleton and collect data from motion capture and force plates.
  \item Analyzed the effect of unilateral stiffness on human gait using MATLAB, identifying key trends in joint angles and ground reaction forces.
  \item Identified key areas for quality-of-life improvements in the lab's data collection pipeline, implementing solutions to streamline the process.
  \end{itemize}
}

% Work Experience Section - Resume Version
\newcommand{\WorkExperienceResume}{
  \sectionheading{Work Experience}
  \entry{Draper Laboratory, Cambridge, MA}{Aug 2022 -- Present}
  \textit{Mechanical Instrumentation Engineer II}
  \begin{itemize}
  \item Developed innovative design features for sensor assemblies, coordinating with cross-functional teams to meet strict requirements.
  \item Designed and tested production prototypes, validating processes for critical assembly features.
  \end{itemize}
  
  \entry{Charge Analytics LLC, Ipswich, MA}{Jun 2021 -- Aug 2021}
  \textit{Design Engineer}
  \begin{itemize}
  \item Executed product development processes, designing electronics packaging and rapid prototyping for low volume production.
  \item Generated CNC fabrication processes, reducing costs by 90\%.
  \end{itemize}
}

% Project Experience Section - Resume Version
\newcommand{\ProjectExperienceResume}{
  \sectionheading{Project Experience}
  \textbf{Robotic Simulation}
  \begin{itemize}
  \item Developed a Furuta pendulum model using Gazebo, C++, and ROS 2, implementing swing-up and energy-minimizing controllers.
  \item Designed and prototyped gearboxes for brushless DC motors, optimizing for assembly and strength.
  \item Created a benchtop dynamometer to evaluate motor performance for a bipedal robot design.
  \end{itemize}
}



