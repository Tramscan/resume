\documentclass[11pt]{article}

% Options for packages loaded elsewhere
\PassOptionsToPackage{unicode}{hyperref}
\PassOptionsToPackage{hyphens}{url}
%

% Set page margins to 0.23 inches
\usepackage[margin=0.23in]{geometry}

% Reduce overall vertical spacing
\usepackage{setspace}
\setstretch{0.9}

% Define a command for section headings with reduced spacing
\newcommand{\sectionheading}[1]{%
  \vspace{0.3em}%
  {\large\textbf{\MakeUppercase{#1}}}\\[-0.3em]%
  \rule{\textwidth}{0.5pt}%
  \vspace{0.3em}%
}

% Define a command for entries with right-justified dates
\newcommand{\entry}[2]{%
  \noindent\makebox[\textwidth]{%
    \parbox{\textwidth}{%
      \textbf{#1}\hfill\textit{#2}%
    }%
  }%
}

% Reduce space between items in lists
\usepackage{enumitem}
\setlist{noitemsep, topsep=0pt, parsep=0pt, partopsep=0pt}

\author{}
\date{}

\begin{document}

% Center, bold, and enlarge name
{\centering
\huge\textbf{Nick Cline}
\par}

\vspace{0.5em}  % Add some vertical space

% Center and reduce spacing for contact info
{\centering
\small
Somerville, MA \textbar{} +1 978-880-3507 \textbar{}
nicholascline1@gmail.com \textbar{} nickcline.com
\par}

\sectionheading{Education}

\entry{University of Massachusetts Amherst Amherst, MA}{May 2022}

\textit{B.S., M.S. in Mechanical Engineering \textbar{} GPA 3.49/4.0}

\textit{Minor in Computer Science}

\sectionheading{Research Experience}

\entry{University of Massachusetts Amherst HRSL, Amherst, MA}{Dec 2021 -- May 2022}
\textit{Graduate Researcher}
\begin{itemize}
\item Created control scripts in python to implement custom impedance control loops for a robotic hip exoskeleton using a Raspberri Pi 4 and the FlexSEA API.
\item Applied asymmetric perturbations to human gait to study behavioral adaptations to a resistant exoskeleton stiffness. Compiled and analyzed data from motion capture and ground reaction forces to identify key trends and possible neuromotor impacts.
\item Used findings to evaluate the gait patterns resulting from the exoskeleton's impedance with results published in IROS 2022.
\item Identified new control schemes for gait control, documenting and testing the initial implementation. Takeaways from the experiments were used to develop tools for realtime data analysis and an improved setup process.
\end{itemize}

\sectionheading{Work Experience}

\entry{Draper Laboratory, Cambridge, MA}{Aug 2022 -- Present}
\textit{Mechanical Instrumentation Engineer II}
\begin{itemize}
\item Developed innovative design features for a sensor assembly by exploring trade spaces and proposing new solutions, coordinating with a cross-functional team to plan reviews, track performance and meet strict requirements.
\item Designed and tested production prototypes, analyzing test data to identify areas for revision and validate production processes for critical assembly features.
\end{itemize}

\entry{Charge Analytics LLC, Ipswich, MA}{Jun 2021 -- Aug 2021}
\textit{Design Engineer}
\begin{itemize}
\item Executed end-to-end product development processes, designing electronics packaging and rapid prototyping for low volume production. Generated CNC fabrication processes to reduce procurement and manufacturing costs by 90%.
\end{itemize}

\sectionheading{Project Experience}

\textbf{Robotic Simulation}
\begin{itemize}
\item Built a Furuta pendulum model for use in Gazebo, controlled with C++ scripts and ROS 2. Developed swing-up and energy-minimizing controllers to stabilize pendulum. Future implementations include RL control of the pendulum and double pendulum control.
\item Designed and prototyped cycloidal and planetary gearboxes for use with brushless DC motors. Tested prototypes using speed control and iterated on design to improve assembly and strength.
\item Produced a benchtop dynamometer prototype to test the two similar DC motors, evaluating performance against cost to optimize motor choice for a bipedal robot design. 
\end{itemize}

\sectionheading{Publications}

Price, M., Abdikadirova, B., Locurto, D., Jaramillo, J. M., \textbf{Cline, N.}, Hoogkamer, W., \& Huber, M. E. (2022). Unilateral stiffness modulation with a robotic hip exoskeleton elicits adaptation during gait. In 2022 IEEE/RSJ International Conference on Intelligent Robots and Systems (IROS) (pp. 12275–12281). 2022 IEEE/RSJ International Conference on Intelligent Robots and Systems (IROS). IEEE. https://doi.org/10.1109/iros47612.2022.9981067 

\sectionheading{Skills \& Interests}
\textbf{Skills:} Atlassian Suite, C, C++, Python, Rust, Julia, MATLAB, Solidworks, ROS 2, Gazebo, Drake, MPC, Nonlinear Dynamics, OpenCV.

\textbf{Interests:} Robotics, Control Theory, Machine Learning, AI, Mechanism Design, 3D Printing, Gardening, Cooking, Weightlifting.

\end{document}
