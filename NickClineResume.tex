\documentclass[11pt]{article}

% Options for packages loaded elsewhere
\PassOptionsToPackage{unicode}{hyperref}
\PassOptionsToPackage{hyphens}{url}
%

% Set page margins to 0.23 inches
\usepackage[margin=0.23in]{geometry}

\author{}
\date{}

\begin{document}

% Center, bold, and enlarge name
{\centering
\huge\textbf{Nick Cline}
\par}

\vspace{0.5em}  % Add some vertical space

% Center and reduce spacing for contact info
{\centering
\small
Somerville, MA \textbar{} +1 978-880-3507 \textbar{}
nicholascline1@gmail.com \textbar{} nickcline.com
\par}

\vspace{1em}  % Add more vertical space before next section

\textbf{Education}

\textbf{University of Massachusetts Amherst Amherst, MA}

\emph{B.S., M.S. in Mechanical Engineering \textbar{} GPA 3.49/4.0 May
2022}

\emph{Minor in Computer Science}

\textbf{Work Experience}

\textbf{Draper Laboratory} \textbf{Cambridge, MA}

\emph{Mechanical Instrumentation Engineer II Aug 2022 -- Present}

\begin{itemize}
\item
  Developed innovative design features for a sensor assembly by
  exploring trade spaces and proposing new solutions, coordinating with
  a cross-functional team to plan reviews, track performance and meet
  strict requirements.
\item
  Designed and tested production prototypes, analyzing test data to
  identify areas for revision and validate production processes for
  critical assembly features.
\end{itemize}

\textbf{Charge Analytics LLC} \textbf{Ipswich, MA}

\emph{Design Engineer Jun 2021 - Aug 2021}

\begin{itemize}
\item
  Executed end-to-end product development processes, from initial
  concept ideation and mechanical design to electronics packaging and
  prototype fabrication using 3D printing techniques.
\item
  Generated CNC Fabrication process for retrofitting low-volume IoT
  parts to reduce costs by nearly 90\% relative to quotes.
\end{itemize}

\textbf{Research Experience}

\textbf{University of Massachusetts Amherst HRSL} \textbf{Amherst, MA}

\emph{Graduate Researcher Dec 2021 - May 2022}

\begin{itemize}
\item
  Created control scripts in Python to implement custom impedance
  control loops on the HRSL Hip Exoskeleton using a Raspberry Pi 4 and
  the FlexSEA API.
\item
  Analyzed the effect of unilateral stiffness on human gait by comparing
  kinematic metrics over stiffness ON and OFF phases, with results
  published in IROS 2022.
\item
  Identified key areas for quality-of-life improvements and wrote
  software solutions to deliver key test metrics in real-time with low
  latency, enabling better feedback for control and evaluation.
\end{itemize}

\textbf{Project experience}

\textbf{Robotic Simulation}

\begin{itemize}
\item
  Built a Furuta pendulum model for use in Gazebo, controlled with C++
  scripts and ROS 2. Developed swing-up and energy-minimizing
  controllers to stabilize pendulum. Future implementations include RL
  control of the pendulum and double pendulum control.
\end{itemize}

\textbf{Publications}

M. Price et al., "Unilateral stiffness modulation with a robotic hip
exoskeleton elicits adaptation during gait," 2022 IEEE/RSJ International
Conference on Intelligent Robots and Systems (IROS), Kyoto, Japan, 2022,
pp. 12275-12281, doi: 10.1109/IROS47612.2022.9981067.

\textbf{Skills \& Interests}

\textbf{Skills:} Atlassian Suite, C, C++, Python, Rust, Julia, MATLAB,
Solidworks, ROS 2, Gazebo, Drake, MPC, Nonlinear Dynamics, OpenCV.

\end{document}
